\chapter{Conclusions and future work}

%% ~5-10 pages

%% - State what you've done and what you've found
%% - Summarize contributions (achievements and impact)
%% - Outline open issues/directions for future work

\section{Comparison to existing models}

\section{Contributions}

??? we made a speech recognition thing

??? we made a synthesizer

??? we made a neural control method for synthesizers

??? we put them together

??? mostly, we've integrated existing parts in a large scale model

\subsection{Contributions to computer science}

Mostly we've talked about brain modeling,
arguably part of the domain of neuroscience,
and control, which is traditionally an engineering topic.
But, this is a CS thesis, so there should be some CS contributions.

\section{Predictions}

need more cortex for consonants than vowels?

- lots of people talk about spectro-temporal features,
  gabor filters across time and space, etc.
  Those are useful models, but not directly implementable
  in neurons; we have to have everything available at
  the same timestep, so derivatives make more sense

\section{Future work}

Summary of things from other sections:

\subsection{Recognition system}

How to deal with semivowels, semiconsonants and glides?
Should it just be one monolithic phoneme detector?

Represent prosody in a second feature layer (hierarchical organization).

Learn all of this stuff rather than optimizing for it.

How to set baseline pitch and volume on the fly?

Can we use artificial cochleas as is?

\subsection{Synthesis system}

We can use this synthesis system to explore
important phonological questions.
For example, syllabic consonants
Do they sound right as separate syllables?
Or do they sound right as protracted versions
of the analogous syllable with the vowel included?
Can these be distinguished from one another?

\chapter{Background}

\section{Human speech}

Make sure to mention monophthong v. diphthong

\section{Auditory periphery modeling}

??? figure like izhikevic, with auditory model + efficiency?

??? summary table with phenomena captured, etc

\section{Automatic speech recognition}

\section{Speech synthesis}

\subsection{Articulatory speech synthesis}

??? summary table with different vocal tract models

??? summary table with different acoustic models

??? in the tables, also note available implementations
(so we can justify writing our own).
Include programming language in this

??? include online vs batch in table

\section{Speech motor control}

??? note that many art. synths consider this part of their
synthesizer (control model).
But we will consider it separately because
it is a primary contribution of this thesis.

\section{Integration}

\section{Vector symbolic architectures}

??? give general background for the math in methods?

%% ~8-20 pages

%% - More than a literature review
%% - Organize related work - impose structure
%% - Be clear as to how previous work being described relates to your own.
%% - The reader should not be left wondering why you've described something!!
%% - Critique the existing work - Where is it strong where is it weak?
%%   What are the unreasonable/undesirable assumptions?
%% - Identify opportunities for more research (i.e., your thesis).
%%   Are there unaddressed, or more important related topics?
%% - After reading this chapter, one should understand the motivation for
%%   and importance of your thesis
%% - You should clearly and precisely define all of the key concepts
%%   dealt with in the rest of the thesis, and teach the reader what s/he
%%   needs to know to understand the rest of the thesis.

\renewcommand\contentsname{Table of Contents}
\tableofcontents
\cleardoublepage
\phantomsection

\addcontentsline{toc}{chapter}{List of Tables}
\listoftables
\cleardoublepage
\phantomsection

\addcontentsline{toc}{chapter}{List of Figures}
\listoffigures
\cleardoublepage
\phantomsection

\chapter*{Typographical Conventions}
\addcontentsline{toc}{chapter}{Typographical Conventions}
\label{typography}

This thesis uses the International Phonetic Alphabet (IPA)
to denote utterance pronunciations.
As is common practice in linguistics,
we will use square brackets (e.g., \ipa{[$\cdot$]})
to denote phonetic transcriptions
(i.e., the actual phones uttered)
and slashes (e.g., \ipa{/$\cdot$/})
to denote phonemic transcriptions
(i.e., the phonemes of interest).
As little attention is paid to prosodic elements like stress,
phone and phoneme strings will be presented
without stress markers.

\cleardoublepage
\phantomsection

% Change page numbering back to Arabic numerals
\pagenumbering{arabic}

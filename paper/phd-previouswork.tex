\chapter{Previous work}

In making early steps toward
an integrated speech recognition and synthesis system,
we are applying techniques
in artificial intelligence and control theory
to the speech domain reviewed in the previous section.
There is a long history of applying these techniques
to this domain;
in this section we review prior modeling efforts
and contrast them with the model we will describe
in future chapters.

\section{Auditory periphery models}

??? figure like izhikevic, with auditory model + efficiency?

??? summary table with phenomena captured, etc

??? Mention artificial neuromorphic cochleas

\subsection{Automatic speech recognition with auditory periphery models}

Another aspect of modern ASR
that does not match biology
is the use of ideal frequency analysis
to generate the feature vectors
which serve as the input to ASR systems.

??? P. 78 of Kollmeier: evidence of amplitude modulation senstivity
in the auditory brain

- p. 67, Kollmeier: expliain modulation filterbank,
  as this seems like it is an important part of the model...

- link that to spectro-temporal receptive fields

??? biologically inspired stuff

\section{Speech synthesis}

\subsection{Articulatory speech synthesis}

??? summary table with different vocal tract models

??? summary table with different acoustic models

??? in the tables, also note available implementations
(so we can justify writing our own).
Include programming language in this

??? include online vs batch in table

\subsection{Speech motor control}

??? note that many art. synths consider this part of their
synthesizer (control model).
But we will consider it separately because
it is a primary contribution of this thesis.

??? Saltzman stuff on task dynamics
is highly related to what we want to do.
But with SPA stuff on top.

??? also hosung nam's work

??? also mention near the end that people haven't
yet connected the Saltzman task dynamics stuff
to the brain; that'll be one of our contributions
